\documentclass[12pt,german]{article}
\usepackage[paper=a4paper,left=24mm,right=24mm,top=20mm,bottom=20mm]{geometry}
\usepackage[utf8]{inputenc}
\usepackage[T1]{fontenc}
\usepackage{babel}
\usepackage{setspace}
\usepackage{booktabs}
\usepackage{amsmath}
\usepackage{graphicx}
\usepackage{subcaption}
\usepackage{setspace}
\usepackage{float}
\usepackage{hyperref}

\usepackage{amsmath}
\usepackage{graphicx}
\usepackage{subcaption}
\usepackage{setspace}
\usepackage{listings}
\usepackage{color}

% \usepackage[english, ngerman]{babel}
% \usepackage{amsmath}                                % For textit, textbf, etc
% \usepackage{graphicx}                               % For includegraphics
% \usepackage{tikz}									% For drawing graphics
% \usetikzlibrary{shapes, backgrounds,mindmap, trees}
% \usepackage{wrapfig}                                % For wrapfig environment
% \usepackage{paralist}                               % For compactitem
% \usepackage{fontawesome}							% For iconsupport
% \usepackage{tabularx}
% \usepackage{array}

%\setcounter{tocdepth}{1} % Show sections
%\setcounter{tocdepth}{2} % + subsections
\setcounter{tocdepth}{3} % + subsubsections
%\setcounter{tocdepth}{4} % + paragraphs
%\setcounter{tocdepth}{5} % + subparagraphs

\definecolor{dkgreen}{rgb}{0,0.6,0}
\definecolor{gray}{rgb}{0.5,0.5,0.5}
\definecolor{mauve}{rgb}{0.58,0,0.82}
\definecolor{gray}{rgb}{0.4,0.4,0.4}
\definecolor{darkblue}{rgb}{0.0,0.0,0.6}
\definecolor{lightblue}{rgb}{0.0,0.0,0.9}
\definecolor{cyan}{rgb}{0.0,0.6,0.6}
\definecolor{darkred}{rgb}{0.6,0.0,0.0}
\definecolor{green}{rgb}{0,0.5,0}
\definecolor{red}{rgb}{0.9,0,0}

\lstset{
  basicstyle=\ttfamily\footnotesize,
  columns=fullflexible,
  showstringspaces=false,
  numbers=left,                   % where to put the line-numbers
  numberstyle=\tiny\color{gray},  % the style that is used for the line-numbers
  stepnumber=1,
  numbersep=5pt,                  % how far the line-numbers are from the code
  backgroundcolor=\color{white},  % choose the background color. You must add \usepackage{color}
  showspaces=false,               % show spaces adding particular underscores
  showstringspaces=false,         % underline spaces within strings
  showtabs=false,                 % show tabs within strings adding particular underscores
  frame=none,                     % adds a frame around the code
  rulecolor=\color{black},        % if not set, the frame-color may be changed on line-breaks within not-black text (e.g. commens (green here))
  tabsize=2,                      % sets default tabsize to 2 spaces
  captionpos=b,                   % sets the caption-position to bottom
  breaklines=true,                % sets automatic line breaking
  breakatwhitespace=false,        % sets if automatic breaks should only happen at whitespace
  title=\lstname,                 % show the filename of files included with \lstinputlisting;
                                  % also try caption instead of title  
  commentstyle=\color{gray}\upshape
}

\lstdefinelanguage{XML}
{
  morestring=[s][\color{mauve}]{"}{"},
  morestring=[s][\color{black}]{>}{<},
  morecomment=[s]{<?}{?>},
  morecomment=[s][\color{dkgreen}]{<!--}{-->},
  stringstyle=\color{black},
  identifierstyle=\color{lightblue},
  keywordstyle=\color{red},
  morekeywords={xmlns,xsi,noNamespaceSchemaLocation,type,id,x,y,source,target,tool,transRef,roleRef,objective,eventually}% list your attributes here
}

\lstdefinelanguage{JAVA}{
  % breakatwhitespace=true,
  keywords={private,public,class,extends,implements,Override,protected,Object,dev,swt,gui,java,util,return,ListResourceBundle}
  keywordstyle=\color{mauve}\bfseries,
  ndkeywords={this,import,package},
  ndkeywordstyle=\color{red}\bfseries,
  comment=[l]{//},
  morecomment=[s]{/*}{*/},
  commentstyle=\color{green},
  keywordstyle=\color{darkblue},
  stringstyle=\color{red},
  basicstyle=\ttfamily,
  moredelim=[il][\textcolor{pgrey}]{$$},
  moredelim=[is][\textcolor{pgrey}]{\%\%}{\%\%}
}

\renewcommand*{\lstlistlistingname}{Auflistungen}
\renewcommand*{\lstlistingname}{Auflistung}

\renewcommand{\labelitemi}{$\bullet$}
\renewcommand{\labelitemii}{$\bullet$}
\renewcommand{\labelitemiii}{$\bullet$}
\renewcommand{\labelitemiv}{$\bullet$}


